\documentclass[a4,12pt]{scrartcl}

%Basic 
\usepackage[utf8]{inputenc}
\usepackage[ngerman]{babel}
\usepackage[T1]{fontenc}
\usepackage{float}
\usepackage[bottom = 3.50cm]{geometry}

%Titel Seite
\title{CLOUD INFRASTRUCTURE}
\subtitle{Lab-07}
\author{Giorgio Vincenti \and Samuel Krieg}
\date{\today}


%Kopf, Fusszeile
\usepackage{fancyhdr}
\pagestyle{fancy}
\lhead{ \begin{picture}(0,0) \put(0,0){\includegraphics[width=3cm]{./pictures/hsrlogo.png}} \end{picture}}
\chead{}
\rhead{Seite \thepage}
\lfoot{Cloud Infrastructure \\Lab-07}
\cfoot{Giorgio Vincenti \and Samuel Krieg}
\rfoot{\today}
\renewcommand{\headrulewidth}{0.4pt}

%Bilder
\usepackage{graphicx}

%Tabellen
\usepackage{booktabs}

%Codesnippets
\usepackage{listings}
\lstset{language=bash} 

%Querformat für eine Seite
\usepackage{lscape}
\usepackage{rotating}
\usepackage{pdflscape}

%Temp
\usepackage{lipsum}



\begin{document}

\clearpage\maketitle
\thispagestyle{empty}
\tableofcontents
\newpage

\section{Anforderungen an ein modernes Datacenter}
\subsection{Herausforderungen}
Die Herausforderungen in einem moderem Datacenter bestehen darin die riesigen Datenmengen, die generiert werden, effizient und schnell verarbeiten zu können. Die Storages und Server Performance spielen dabei eine grosse rolle, jedoch bringt diese Performance nichts, wenn das Netzwerk nicht dementsprechend mit den Datenmengen effizient umgehen kann. Die grossen Datenmengen verlassen das Datacenter nicht, sondern fliessen von System zu System.

\subsubsection{Beispiele}
Hier einige Beispiele in der Praxis. \\

\noindent \textbf{VMotion:} Zum Beispiel bei Einsatz von Virtualisierung muss der Administrator in der Lage sein, VMs über das Netzwerk auf andere Hosts innerst kürzester Zeit verschieben können. \\

\noindent \textbf{Webserver:} Ein anderes Beispiel bezieht sich auf ein Webserver/Datenbank Modell. Der Webserver muss Daten der Datenbank abfragen und abholen ohne grossen Verzögerungen.\\

\noindent \textbf{Backup:} Ein weiteres Beispiel betrifft die Backups der Maschinen. Da fliessen ebenfalls grosse Datenmengen übers Netzwerk. Falls mehrere Server innerhalb eines gewissen Zeitraums gesichert werden müssen, muss die Netzwerkperformance stimmen, damit keine Engpässe auftreten. 

\subsection{Anforderungen}
Die Anforderungen an ein solches Datacenter sind sehr hoch. Da gibt es einiges zu beachten: 
\begin{itemize}
\item Sehr hohe Verfügbarkeit 
\item Mobilität von Applikationen, Systemen und Daten
\item Zugriffe auf Applikationen, Systemen und Daten 
\item Hohe Auto-Skalierbarkeit 
\item Genaue Abrechnungen auf benutzte Dienste 
\item Sicherheit 
\item Hohe Performance 
\item Abhängigkeit 
\item Autoprovisioning von Applikationen, Systemen und Daten
\end{itemize}

\subsection{Netzwerkarchitektur}
Die Netzwerkarchitektur spielt eine sehr grosse Rolle in einem Datacenter. Auch diese bringt grosse Anforderungen mitsich, damit das Beste aus der Infrastruktur rausgeholt werden kann. 
\begin{itemize}
\item Skalierbarkeit
\item Performance 
\item Hohen Datendurchsatz zwischen den Systemen
\item Elephant-flows fähig 
\item Managebar 
\end{itemize}

\section{Trill}
\subsection{Function / Protocol header:} 
\subsection{Goal of TRILL:}
\subsection{Why was TRILL developed?}
\subsection{Where and how can TRILL be used?}
\subsection{Is TRILL comparable to a traditional protocol?}
\subsection{What are the advantages and disadvantages of TRILL?}
\subsection{Are there other modern technologies which are similar to TRILL}
\subsection{How is TRILL configured in the LAB?}
\subsubsection{Trees}
\subsubsection{PacketFlow}
\subsubsection{MAC Learning}
\section{VXLAN}
\section{VXLAN with Arista switches}
\section{VXLAN Configuration}
\section{Research modern data center requirements}
Text


\subsection{Subsection}
Text


\subsubsection{Subsubsection}
Text

\section{Aufzählung}
\subsection{Itemize}
\begin{itemize}
\item Das erste Item
\item Das zweite Item
\begin{itemize}
\item Das erste Item
\item Das zweite Item
\item Das dritte Item
\end{itemize}
\item Das dritte Item
\end{itemize}

\subsection{Enumerate}
\begin{enumerate}
  \item The first item
  \item The second item
  \item The third etc \ldots
\end{enumerate}

\subsection{Description}
\begin{description}
  \item[First] The first item
  \item[Second] The second item
  \item[Third] The third etc \ldots
\end{description}
\begin{description}
  \item[First] \hfill \\
  The first item
  \item[Second] \hfill \\
  The second item
  \item[Third] \hfill \\
  The third etc \ldots
\end{description}

\section{Tabellen}
\begin{center}
    \begin{tabular}{@{} l l r@{}}\toprule    
    {Stockwerk} & {Hostname} & {Anzahl Ports}\\ \midrule
    1 & DataCSW01 & 12\\ \addlinespace
    & DataCSW03 & 12\\ \addlinespace
    & DataCSW04 & 12\\ \addlinespace
    2& DataCSW02 & 24\\
    \bottomrule
    \end{tabular}
\end{center}

\section{Codesnippets}
\begin{lstlisting}
sudo apt-get install qemu
\end{lstlisting}

\end{document}
